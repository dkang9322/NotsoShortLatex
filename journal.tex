%preamble begin
\documentclass[a4paper, 11pt]{article}

\author{H.~Partl}
\title{Minimalism}

%body begin
\begin{document}
%generates the title
\maketitle
%insert the table of contents
\tableofcontents

\section{Some Interesting Words}
Well, and here begins my article
\section[ShortTitle]{TooLong to be included in the table of contents %
so I'll cut it short}
\ldots{} and here it ends %note purposeful space after ...
\section{CrossReferencing}
\ldots this is a reference to %
\label{myLabel} %
the current section \ref{myLabel} %
on page \pageref{myLabel} \ldots

\section{Itemize and Enumerate Mixed}
\flushleft %left flush
\begin{enumerate}
% item for enum
\item You can nest the list environments as you like:

  \begin{itemize}
  %item for itemize
  \item But it might start to look silly.
  \item[-] With a dash.
  \end{itemize}

\item Therefore remember:

  \begin{description}
  \item[Stupid] things will not become smart because they%
    are in a list
  \item[Smart] things, though, can be presented%
    beautifully in a list.
  \end{description}
\end{enumerate}

\section{Quotes}
A typographical rule of thumb for the line length is:

\begin{quote}
  On average, no line should be longer than 66 characters.
\end{quote}

% Don't forget the curly braces to have a space after LaTeX
This is why \LaTeX{} pages have such large borders by default and %
also why multicolumn print is used in newspapers.
\end{document}
